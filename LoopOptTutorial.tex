% !TeX encoding = UTF-8
% !TeX program = lualatex
% !TeX spellcheck = en_US

% An alternative to the following \documentclass is
% \documentclass{beamer}
% \mode<presentation>
% {
%   \usetheme{default}
%   \setbeamercovered{transparent}
% }

\documentclass[presentation,english,aspectratio=169]{beamer}

% Remove the annoying symbols no one uses anyway.
\beamertemplatenavigationsymbolsempty

% Most of these were autogenerated by org export to beamer.tex file
% I don't know if they are all necessary, but I don't think there is any harm in
% including them.
\usepackage[utf8]{luainputenc}
%\usepackage[T1]{fontenc} %MK: LuaLaTeX only properly supports unicode fonts ("TU")

\usepackage{graphicx}
\usepackage{grffile}
\usepackage{longtable}
\usepackage{wrapfig}
\usepackage{rotating}
\usepackage[normalem]{ulem}
\usepackage{amsmath}
\usepackage{textcomp}
\usepackage{amssymb}
\usepackage{capt-of}
\usepackage{hyperref}
\usetheme{default}

% I'm not sure what exactly this is for
% It is copied from a template provided by Till Tantau
% <tantau@users.sourceforge.net>
%\usepackage[english]{babel} %MK: for LuaLaTeX, use polyglossia (if needed)

% It is copied from a template provided by Till Tantau
% <tantau@users.sourceforge.net>
% I'm not sure what exactly this is for
%\usepackage{times} %MK: Not available for unicode fonts (it's a Type-1 font, obsolete and its maintainer is dead); use \uspackage{fontspec} to customize font.

\usepackage{listings}

\usepackage{minibox}
\usepackage{tikz}
\usetikzlibrary{decorations.text}
\usetikzlibrary{shadows}
\usetikzlibrary{fit}
\usetikzlibrary{graphs}
\usetikzlibrary{graphdrawing}
\usegdlibrary{layered}

\pgfdeclarelayer{backbackbackground}
\pgfdeclarelayer{backbackground}
\pgfdeclarelayer{background}
\pgfdeclarelayer{foreground}
\pgfdeclarelayer{foreforeground}
\pgfsetlayers{backbackbackground,backbackground,background,main,foreground,foreforeground}


% Use this to adjust text for overlays
% If set to transparent, then text is overlays is visible, but greyed out.
% If not set, then the text in overlays is not visible at all (i.e., invisible).
\setbeamercovered{transparent}

\usecolortheme{}
\usefonttheme{}
\useinnertheme{}
\useoutertheme{}
\author[Author, Another] % {optional, use only with lots of authors
       {Kit Barton\inst{1} \and \\
        Johannes Doerfert\inst{2} \and \\
        Hal Finkel\inst{2} \and \\
        Michael Kruse\inst{2} \and \\
        Ettore Tiotto\inst{1}}

% Give the names in the same order as they appear in the paper.
% Use the \inst{?} command only if the authors have different
% affiliation.
\institute[IBM Canada and ANL] % (optional, but mostly needed)
{
  \inst{1}
  IBM Canada
  \inst{2}
  Argonne National Laboratory
}

% Pick the date.
% Hard code it or pick a floating date based on day the PDF was built.
% Can also be a string
%\date[August 15, 2019]{Name of Conference here}
\date{October 23, 2019}

\title{Writing Loop Optimizations in LLVM}

% Autogenerated by org export to beamer.tex file
\hypersetup{
 pdfauthor={Kit Barton},
 pdftitle={Beamer Template},
 pdfkeywords={},
 pdfsubject={},
 pdfcreator={},
 pdflang={English}}

%\subject{Theoretical Computer Science}
% This is only inserted into the PDF information catalog. Can be left
% out.



% If you have a file called "university-logo-filename.xxx", where xxx
% is a graphic format that can be processed by latex or pdflatex,
% resp., then you can add a logo as follows:

\pgfdeclareimage[height=0.5cm]{ibm-logo}{IBMLogo.png}
\pgfdeclareimage[height=0.5cm]{anl-logo}{anl-symbol} % MK: Feel free to replace with smaller logo
\logo{\pgfuseimage{anl-logo}\hspace*{.9\linewidth}\pgfuseimage{ibm-logo}}

% Delete this, if you do not want the table of contents to pop up at
% the beginning of each subsection:
%% \AtBeginSubsection[]
%% {
%%   \begin{frame}<beamer>{Outline}
%%     \tableofcontents[currentsection,currentsubsection]
%%   \end{frame}
%% }


% If you wish to uncover everything in a step-wise fashion, uncomment
% the following command:

%\beamerdefaultoverlayspecification{<+->}



\makeatletter
% pgf 'file' shape
\def\myfoldheight{0.5}
\def\myshapepath{
	\pgfextract@process\northwest{
		\southwest\pgf@xa=\pgf@x
		\northeast
		\pgf@x=\pgf@xa
	}

	\pgfextract@process\southeast{
		\southwest\pgf@ya=\pgf@y
		\northeast
		\pgf@y=\pgf@ya
	}

	\pgfextract@process\northfold{
		\pgfpointdiff{\southwest}{\northeast}
		\northeast
		\advance\pgf@x-\myfoldheight\pgf@y
	}

	\pgfextract@process\eastfold{
		\pgfpointdiff{\southwest}{\northeast}
		\northeast
		\advance\pgf@y-\myfoldheight\pgf@y
	}

	\pgfextract@process\fold{
		\northfold\pgf@xa=\pgf@x
		\eastfold
		\pgf@x=\pgf@xa
	}

	\pgfpathmoveto{\southwest}
	\pgfpathlineto{\northwest}
	\pgfpathlineto{\northfold}
	\pgfpathlineto{\eastfold}
	\pgfpathlineto{\southeast}
	\pgfpathclose
}

% compute an intersection point between a line and \myshapepath
% NOTE: Breaks inside \graph[layered layout]
\def\myshapeanchorborder#1#2{
	% #1 = point inside the shape
	% #2 = direction
	\pgftransformreset % without this, the intersection commands yield strange results
	\pgf@relevantforpicturesizefalse % don't include drawings in bounding box
	\pgfintersectionofpaths{
		\myshapepath
		%\pgfgetpath\temppath\pgfusepath{stroke}\pgfsetpath\temppath % draw path for debugging
	}{
		\pgfpathmoveto{
			\pgfpointadd{
				\pgfpointdiff{\southwest}{\northeast}\pgf@xc=\pgf@x \advance\pgf@xc by \pgf@y % calculate a distance that is guaranteed to be outside the shape
				\pgfpointscale{
					\pgf@xc
				}{
					\pgfpointnormalised{
						#2
					}
				}
			} {
				#1
			}
		}
		\pgfpathlineto{#1}
		%\pgfgetpath\temppath\pgfusepath{stroke}\pgfsetpath\temppath % draw path for debugging
	}
	\pgfpointintersectionsolution{1}
}
\def\myshapeanchorcenter{
	\pgfpointscale{.5}{\pgfpointadd{\southwest}{\northeast}}
}

% we could probably re-use some existing \dimen, but better be careful
\newdimen\myshapedimenx
\newdimen\myshapedimeny

\pgfdeclareshape{file}{
	% some stuff, we can inherit from the rectangle shape
	\inheritsavedanchors[from=rectangle]
	\inheritanchor[from=rectangle]{center}
	\inheritanchor[from=rectangle]{mid}
	\inheritanchor[from=rectangle]{base}

	% calculate these anchors so they lie on a coorinate line with .center
	\inheritanchor[from=rectangle]{west}
	\inheritanchor[from=rectangle]{east}
	\inheritanchor[from=rectangle]{north}
	\inheritanchor[from=rectangle]{south}

	% calculate these anchors so they lie on a line through .center and the corresponding anchor of the underlying rectangle
	\inheritanchor[from=rectangle]{south west}
	\inheritanchor[from=rectangle]{south east}
	\inheritanchor[from=rectangle]{north west}
	%    \inheritanchor[from=rectangle]{north east}

	% somewhat more special anchors. The coordinate calculations were taken from the rectangle node
	\inheritanchor[from=rectangle]{mid west}
	\inheritanchor[from=rectangle]{mid east}
	\inheritanchor[from=rectangle]{base west}
	\inheritanchor[from=rectangle]{base east}

	\backgroundpath{
		\myshapepath
	}

	\foregroundpath{
		\pgfpathmoveto{\northfold}
		\pgfpathlineto{\fold}
		\pgfpathlineto{\eastfold}
	}

	% This is from rectangle, i.e. without the fold.
	\anchorborder{%
		\pgf@xb=\pgf@x% xb/yb is target
		\pgf@yb=\pgf@y%
		\southwest%
		\pgf@xa=\pgf@x% xa/ya is se
		\pgf@ya=\pgf@y%
		\northeast%
		\advance\pgf@x by-\pgf@xa%
		\advance\pgf@y by-\pgf@ya%
		\pgf@xc=.5\pgf@x% x/y is half width/height
		\pgf@yc=.5\pgf@y%
		\advance\pgf@xa by\pgf@xc% xa/ya becomes center
		\advance\pgf@ya by\pgf@yc%
		\edef\pgf@marshal{%
			\noexpand\pgfpointborderrectangle
			{\noexpand\pgfqpoint{\the\pgf@xb}{\the\pgf@yb}}
			{\noexpand\pgfqpoint{\the\pgf@xc}{\the\pgf@yc}}%
		}%
		\pgf@process{\pgf@marshal}%
		\advance\pgf@x by\pgf@xa%
		\advance\pgf@y by\pgf@ya%
	}

	%    \anchorborder{
	%        \myshapedimenx=\pgf@x
	%        \myshapedimeny=\pgf@y
	%        \myshapeanchorborder{\myshapeanchorcenter}{\pgfpoint{\myshapedimenx}{\myshapedimeny}}
	%    }
}

\newsavebox{\my@resizeenv@TempBox}%
\newcommand*{\my@resizeenv@width}{}%
\newenvironment{resizeenv}[1]{%
\renewcommand*{\my@resizeenv@width}{#1}%
\begin{lrbox}{\my@resizeenv@TempBox}%
}{%
\end{lrbox}%
\resizebox{\my@resizeenv@width}{!}{\usebox{\my@resizeenv@TempBox}}%
}%

\newenvironment{resizepar}{%
\begin{resizeenv}{\textwidth}%
}{%
\end{resizeenv}%
}%
\makeatother


\begin{document}

% Create the title page
\begin{frame}
  \titlepage
\end{frame}

% Create an outline slide
\begin{frame}{Outline}
  \tableofcontents
  % You might wish to add the option [pausesections]
\end{frame}

% Create a section; comment out if not desired.
\section{Terminology}
\label{sec:terminology}

\begin{frame}[label={sec:org8787e08}]{Loop Representation in LLVM}

  % KB: I will clean this slide up and make it into a proper beamer slide, with
  % animations, if everyone is OK with how it looks.
  \includegraphics[width=\textwidth]{LoopRepresentation.png}
\end{frame}

\begin{frame}[label={sec:orgac3eb21}]{Other Terminology}
  \begin{block}{}
    \begin{itemize}
    \item Loop predecessor
    \item Backedge taken count
    \item Iteration count
    \item Loop guard
    \item irreducible loops (not covered)
    \item Others?
    \end{itemize}
  \end{block}
\end{frame}

\begin{frame}[label={sec:rotated}]{Rotated Loops}
  \begin{itemize}
    \item Describe what rotated loops are
    \item Give some examples of rotated loops
    \item Limitations of loop rotation
  \end{itemize}
\end{frame}

\begin{frame}[label={sec:canonical}]{Canonical Form}
Describe loop canonical form - what exactly this entails, \textit{etc}.
\end{frame}

\begin{frame}[label={sec:lcssa}]{Loop Closed SSA Form}
% MK: TODO
    What exactly is this? When is it useful? What are the implications afterwards?
\end{frame}

%MK: Not sure whether I am going to need this; for now it's for testing whether tikzpictruew works
\begin{frame}[fragile]{Clang/LLVM/Polly Compiler Pipeline}
\begin{resizepar}
\begin{tikzpicture}
\tikzset{tight/.style={inner sep=0pt,outer sep=0pt,minimum size=0pt}}
\tikzset{node/.style={draw=teal,line width=1.2pt,rounded corners,top color=teal!50,bottom color=white,shading angle=15,drop shadow}}
\tikzset{supernode/.style={subgraph text none,bottom color=teal!30,top color=teal!05,shading angle=15,rounded corners}}
\tikzset{edge/.style={->}}

\graph[layered layout,edges={edge,rounded corners},level sep=5mm,sibling sep=10mm]{
	c[as={\minibox{\texttt{void f() \{}\\\hspace*{4mm}\texttt{for (int i=...)}\\\hspace*{4mm}\dots}},grow=right,draw,shape=file,fill=white,label={source.c}];
	ir[as={IR},shape=file,draw,fill=white,nudge=(up:10mm)];
	asm[as={Assembly},shape=file,draw,fill=white];

	clang [subgraph text none,label={[font=\Large\sffamily]above:Clang}] // [sibling sep=2mm,grow=down,layered layout] {
		lexer [as={Lexer},node];
		parser [as={Parser},node,grow=down];
		preprocessor [as={Preprocessor},node];
		sema [as={Semantic Analyzer},node];
		codegen [as={IR Generation},node];
		%
		lexer->preprocessor->parser->sema->codegen;
	};

	llvm [subgraph text none,label={[font=\Large\sffamily]above:LLVM}] // [sibling sep=2mm,grow=down,layered layout] {
		canonicalization [as={Canonicalization passes},node];
		loopopts [as={Loop optimization passes},node];
		polly [as={Polly},node];
		vectorization [as={LoopVectorize},node];
		latepasses [as={Late Mid-End passes},node];
		backend [as={Target Backend},node];
		%
		canonicalization->loopopts->vectorization->latepasses->backend;
		loopopts->polly->vectorization;
	};

	c->[in=180]lexer;
	codegen->[out=0,in=-90]ir;
	ir->[out=90,in=180]canonicalization;
	backend->[out=0,in=-90]asm;
};

\begin{pgfonlayer}{background}
\node[tight,fit={(clang)},supernode]{};
\node[tight,fit={(llvm)},supernode]{};
\end{pgfonlayer}

%	\path (preprocessor) edge[edge,densely dotted,bend left=50,draw=blue!50!black] node[midway,right,font=\ttfamily,blue!50!black] {\#pragma} (sema);
%	\path (preprocessor) edge[edge,densely dotted,bend right=80,draw=blue!50!black] node[pos=0.7,left,font=\ttfamily,blue!50!black] {\#pragma} (codegen);

	\path (codegen) edge[edge,dashed,bend right=80,draw=blue!50!black,postaction={decorate,decoration={text along path,text={|\color{blue!50!black}|Loop metadata},raise=-1.7ex,pre=moveto,pre length=13mm}}] (ir);
	\path (ir) edge[edge,dashed,draw=blue!50!black] (loopopts);
%	\path (ir) edge[edge,dashed,draw=blue!50!black] (polly);1
	\path (ir) edge[edge,dashed,draw=blue!50!black,bend right=10] (vectorization);
%	\path (polly) edge[edge,dashed,draw=blue!50!black,bend right=10] (vectorization);
\end{tikzpicture}
\end{resizepar}
\end{frame}


\begin{frame}{Current LLVM Loop Optimization Passes}
\end{frame}

\begin{frame}[label={sec:auxdatastruct}]{Auxiliary Data Structures}
    \begin{itemize}
    \item Dominator/PostDominator trees
    \item Data dependence graph (DDG)
    \item Others??
    \end{itemize}
\end{frame}

\section{Considerations when writing a loop pass}
\begin{frame}[label={sec:looppass}]{Loop Pass vs Function Pass}
  \begin{itemize}
  \item What is a loop pass
  \item What is a function pass
  \item Differences/pros/cons of using one over the other
  \end{itemize}
\end{frame}

\begin{frame}[label={sec:passmanager}]{Pass Managers}
  \begin{itemize}
    \item New Pass Manager
    \item Legacy Pass Manager (JD: I would only mention: use the new one)
    \item Where to put loop optimizations in the loop opt pipeline (maybe on new slide)
  \end{itemize}
\end{frame}

\section{Other Useful Tools}
\begin{frame}[label=reports]{Reporting success and failure}
  \begin{itemize}
    \item STATISTICS
    \item Optimization Remark Emitter
  \end{itemize}
\end{frame}

\begin{frame}[label={sec:datastructs}]{Updating Data Structures}
    \begin{itemize}
      \item DominatorTreeUpdater
      \item SCEV
      \item Other??
    \end{itemize}
\end{frame}

\end{document}
