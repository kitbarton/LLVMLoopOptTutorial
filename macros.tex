% !TeX encoding = UTF-8
% !TeX spellcheck = en_US
% !TeX root = LoopOptTutorial.tex

% Remove the annoying symbols no one uses anyway.
\beamertemplatenavigationsymbolsempty

% Most of these were autogenerated by org export to beamer.tex file
% I don't know if they are all necessary, but I don't think there is any harm in
% including them.
\usepackage[utf8]{luainputenc}
%\usepackage[T1]{fontenc} %MK: LuaLaTeX only properly supports unicode fonts ("TU")

\usepackage{graphicx}
%\usepackage{grffile} %MK: https://github.com/ho-tex/oberdiek/issues/73
\usepackage{longtable}
\usepackage{wrapfig}
\usepackage{rotating}
\usepackage[normalem]{ulem}
\usepackage{amsmath}
\usepackage{textcomp}
\usepackage{amssymb}
\usepackage{capt-of}
\usepackage{hyperref}
\usepackage{comment}
\usetheme{default}

% I'm not sure what exactly this is for
% It is copied from a template provided by Till Tantau
% <tantau@users.sourceforge.net>
%\usepackage[english]{babel} %MK: for LuaLaTeX, use polyglossia (if needed)

% It is copied from a template provided by Till Tantau
% <tantau@users.sourceforge.net>
% I'm not sure what exactly this is for
%\usepackage{times} %MK: Not available for unicode fonts (it's a Type-1 font, obsolete and its maintainer is dead); use \uspackage{fontspec} to customize font.

\usepackage{listings}
\usepackage{xcolor}
\usepackage{minibox}
\usepackage{tikz}
\usetikzlibrary{decorations.text}
\usetikzlibrary{shapes.arrows}
\usetikzlibrary{arrows.meta}
\usetikzlibrary{bending}
\usetikzlibrary{shadows}
\usetikzlibrary{fit}
\usetikzlibrary{graphs}
\usetikzlibrary{graphdrawing}
\usetikzlibrary{matrix}
\usetikzlibrary{positioning}
\usegdlibrary{layered}

\pgfdeclarelayer{backbackbackground}
\pgfdeclarelayer{backbackground}
\pgfdeclarelayer{background}
\pgfdeclarelayer{foreground}
\pgfdeclarelayer{foreforeground}
\pgfsetlayers{backbackbackground,backbackground,background,main,foreground,foreforeground}
\tikzset{>={Stealth[round,flex,length=5pt 4.5 0.8]}} % Standard latex arrow tips are way too small
\tikzset{tight/.style={minimum width=0pt,minimum height=0pt,inner sep=0pt,outer sep=0pt}}
\usepackage{aobs-tikz}

\usepackage{minted}
\newmintinline{c}{style=bw}
\newmintinline{cpp}{style=bw}
\newmintinline{llvm}{}
\newmintinline{text}{style=bw}
\usemintedstyle{borland}

\lstset{
  basicstyle=\ttfamily,
  columns=fullflexible,
  breaklines=true,
  postbreak=\raisebox{0ex}[0ex][0ex]{\color{red}$\hookrightarrow$\space}
}

% Use this to adjust text for overlays
% If set to transparent, then text is overlays is visible, but greyed out.
% If not set, then the text in overlays is not visible at all (i.e., invisible).
\setbeamercovered{transparent}

\usecolortheme{}
\usefonttheme{}
\useinnertheme{}
\useoutertheme{}

\setbeamertemplate{blocks}[rounded][shadow=true]
\setbeamercolor{block title}{bg=teal,fg=white}
\setbeamercolor{block body}{bg=teal!10}

% Delete this, if you do not want the table of contents to pop up at
% the beginning of each subsection:
%% \AtBeginSubsection[]
%% {
%%   \begin{frame}<beamer>{Outline}
%%     \tableofcontents[currentsection,currentsubsection]
%%   \end{frame}
%% }


% If you wish to uncover everything in a step-wise fashion, uncomment
% the following command:

%\beamerdefaultoverlayspecification{<+->}



\makeatletter
% pgf 'file' shape
\def\myfoldheight{0.5}
\def\myshapepath{
	\pgfextract@process\northwest{
		\southwest\pgf@xa=\pgf@x
		\northeast
		\pgf@x=\pgf@xa
	}

	\pgfextract@process\southeast{
		\southwest\pgf@ya=\pgf@y
		\northeast
		\pgf@y=\pgf@ya
	}

	\pgfextract@process\northfold{
		\pgfpointdiff{\southwest}{\northeast}
		\northeast
		\advance\pgf@x-\myfoldheight\pgf@y
	}

	\pgfextract@process\eastfold{
		\pgfpointdiff{\southwest}{\northeast}
		\northeast
		\advance\pgf@y-\myfoldheight\pgf@y
	}

	\pgfextract@process\fold{
		\northfold\pgf@xa=\pgf@x
		\eastfold
		\pgf@x=\pgf@xa
	}

	\pgfpathmoveto{\southwest}
	\pgfpathlineto{\northwest}
	\pgfpathlineto{\northfold}
	\pgfpathlineto{\eastfold}
	\pgfpathlineto{\southeast}
	\pgfpathclose
}

% compute an intersection point between a line and \myshapepath
% NOTE: Breaks inside \graph[layered layout]
\def\myshapeanchorborder#1#2{
	% #1 = point inside the shape
	% #2 = direction
	\pgftransformreset % without this, the intersection commands yield strange results
	\pgf@relevantforpicturesizefalse % don't include drawings in bounding box
	\pgfintersectionofpaths{
		\myshapepath
		%\pgfgetpath\temppath\pgfusepath{stroke}\pgfsetpath\temppath % draw path for debugging
	}{
		\pgfpathmoveto{
			\pgfpointadd{
				\pgfpointdiff{\southwest}{\northeast}\pgf@xc=\pgf@x \advance\pgf@xc by \pgf@y % calculate a distance that is guaranteed to be outside the shape
				\pgfpointscale{
					\pgf@xc
				}{
					\pgfpointnormalised{
						#2
					}
				}
			} {
				#1
			}
		}
		\pgfpathlineto{#1}
		%\pgfgetpath\temppath\pgfusepath{stroke}\pgfsetpath\temppath % draw path for debugging
	}
	\pgfpointintersectionsolution{1}
}
\def\myshapeanchorcenter{
	\pgfpointscale{.5}{\pgfpointadd{\southwest}{\northeast}}
}

% we could probably re-use some existing \dimen, but better be careful
\newdimen\myshapedimenx
\newdimen\myshapedimeny

\pgfdeclareshape{file}{
	% some stuff, we can inherit from the rectangle shape
	\inheritsavedanchors[from=rectangle]
	\inheritanchor[from=rectangle]{center}
	\inheritanchor[from=rectangle]{mid}
	\inheritanchor[from=rectangle]{base}

	% calculate these anchors so they lie on a coorinate line with .center
	\inheritanchor[from=rectangle]{west}
	\inheritanchor[from=rectangle]{east}
	\inheritanchor[from=rectangle]{north}
	\inheritanchor[from=rectangle]{south}

	% calculate these anchors so they lie on a line through .center and the corresponding anchor of the underlying rectangle
	\inheritanchor[from=rectangle]{south west}
	\inheritanchor[from=rectangle]{south east}
	\inheritanchor[from=rectangle]{north west}
	%    \inheritanchor[from=rectangle]{north east}

	% somewhat more special anchors. The coordinate calculations were taken from the rectangle node
	\inheritanchor[from=rectangle]{mid west}
	\inheritanchor[from=rectangle]{mid east}
	\inheritanchor[from=rectangle]{base west}
	\inheritanchor[from=rectangle]{base east}

	\backgroundpath{
		\myshapepath
	}

	\foregroundpath{
		\pgfpathmoveto{\northfold}
		\pgfpathlineto{\fold}
		\pgfpathlineto{\eastfold}
	}

	% This is from rectangle, i.e. without the fold.
	\anchorborder{%
		\pgf@xb=\pgf@x% xb/yb is target
		\pgf@yb=\pgf@y%
		\southwest%
		\pgf@xa=\pgf@x% xa/ya is se
		\pgf@ya=\pgf@y%
		\northeast%
		\advance\pgf@x by-\pgf@xa%
		\advance\pgf@y by-\pgf@ya%
		\pgf@xc=.5\pgf@x% x/y is half width/height
		\pgf@yc=.5\pgf@y%
		\advance\pgf@xa by\pgf@xc% xa/ya becomes center
		\advance\pgf@ya by\pgf@yc%
		\edef\pgf@marshal{%
			\noexpand\pgfpointborderrectangle
			{\noexpand\pgfqpoint{\the\pgf@xb}{\the\pgf@yb}}
			{\noexpand\pgfqpoint{\the\pgf@xc}{\the\pgf@yc}}%
		}%
		\pgf@process{\pgf@marshal}%
		\advance\pgf@x by\pgf@xa%
		\advance\pgf@y by\pgf@ya%
	}

	%    \anchorborder{
	%        \myshapedimenx=\pgf@x
	%        \myshapedimeny=\pgf@y
	%        \myshapeanchorborder{\myshapeanchorcenter}{\pgfpoint{\myshapedimenx}{\myshapedimeny}}
	%    }
}

\newsavebox{\my@resizeenv@TempBox}%
\newcommand*{\my@resizeenv@width}{}%
\newenvironment{resizeenv}[1]{%
\renewcommand*{\my@resizeenv@width}{#1}%
\begin{lrbox}{\my@resizeenv@TempBox}%
}{%
\end{lrbox}%
\resizebox{\my@resizeenv@width}{!}{\usebox{\my@resizeenv@TempBox}}%
}%

\newsavebox{\my@scale@Lrbox}
\newcommand*{\my@scale@Percentage}{}
\newenvironment*{scaleenv}[1]{%
\renewcommand*{\my@scale@Percentage}{#1}%
\begin{lrbox}{\my@scale@Lrbox}%
}{%
\end{lrbox}%
\scalebox{\my@scale@Percentage}{\usebox{\my@scale@Lrbox}}%
}%

\newsavebox{\my@scalepar@TempBox}
\newenvironment{scalepar}[1]{%
\def\my@DoScalePar{\scalebox{#1}}%
\begin{lrbox}{\my@scalepar@TempBox}%
\pgfmathparse{\textwidth/#1}%
\begin{minipage}{\pgfmathresult pt}%
}{%
\end{minipage}%
\end{lrbox}%
\my@DoScalePar{\usebox{\my@scalepar@TempBox}}%
}


\newenvironment{resizepar}{%
\begin{resizeenv}{\textwidth}%
}{%
\end{resizeenv}%
}%

\newenvironment{tightcenter}{%
  \setlength\topsep{0pt}
  \setlength\parskip{0pt}
  \begin{center}
}{%
  \end{center}
}

\newcommand{\anchorbox}[3][]{%
   \tikz[remember picture,baseline=(#2.base)] \node[inner sep=0,outer sep=0,minimum size=0pt,#1](#2) {\minibox{#3}};%
}

% Positioniert etwas an eine bestimmte Stelle 
% \begin{locate}[options]{position}
\newsavebox{\my@locate@Lrbox}
\newenvironment*{locate}[2][]{% 
        \tikzset{my@locate@Options/.style={#1}}%
        \begin{tikzpicture}[remember picture,overlay]%
        \coordinate (my@locate@Position) at (#2);
        \end{tikzpicture}%
        \begin{lrbox}{\my@locate@Lrbox}%
}{%
        \end{lrbox}%
        \begin{tikzpicture}[remember picture,overlay]%
        \node[my@locate@Options,inner sep=0,outer sep=0] at (my@locate@Position) {\usebox{\my@locate@Lrbox}};%
        \end{tikzpicture}%
}


\renewenvironment<>{locate}[2][]{%
\begin{actionenv}#3\begin{originallocate}[#1]{#2}%
}{%
\end{originallocate}\end{actionenv}%
}





% Tikz coordinate system: Relativ zur aktuellen Seite
% "page cs:x=0,y=0" == Oben links
% "page cs:x=1,y=1" == Unten rechts
\define@key{pagekeys}{x}{\def\mypagex{#1}}
\define@key{pagekeys}{y}{\def\mypagey{#1}}
\tikzdeclarecoordinatesystem{page}{%
\setkeys{pagekeys}{#1}%
\pgfpointlineattime{%
        \mypagex%
}{%
        \pgfpointlineattime{%
                \mypagey%
        }{%
                \pgfpointanchor{current page}{north west}%
        }{%
                \pgfpointanchor{current page}{south west}%
        }%
}{%
        \pgfpointlineattime{%
                \mypagey%
        }{%
                \pgfpointanchor{current page}{north east}%
        }{%
                \pgfpointanchor{current page}{south east}%
        }%
}%
}

\makeatother

\newcommand*\arrowdown[1]{%
\begin{tikzpicture}%
%\node[text width=\textwidth,outer ysep=0pt,inner ysep=0pt](text){};
\node[draw=teal!90!black,very thick,single arrow,rotate=-90,anchor=center,minimum width=8mm,minimum height=8mm,outer ysep=0pt,rounded corners=1pt,drop shadow,shade,top color=teal!10,bottom color=teal,shading angle=15](arrow) {};
\ifthenelse{\equal{#1}{}}{}{\node[right=3mm] at (arrow.north) {\minibox{#1}};}
%\draw[blue] (current bounding box.north west) rectangle (current bounding box.south east);
\end{tikzpicture}%
}

\definecolor{darkmagenta}{rgb}{0.55, 0.0, 0.55}
\definecolor{darkspringgreen}{rgb}{0.09, 0.45, 0.27}
